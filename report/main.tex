\documentclass{article}

\usepackage{stdheader}
\graphicspath{{../figs/}}

\DeclareMathOperator\Tr{Tr}

\title{the Berry Phase}
\author{Jacob MacWilliams}

\newtheorem{thm}{Theorem}

\begin{document}

\maketitle
\newpage
\tableofcontents
\newpage

\section{Introduction \label{sec:intro}}

It has long been known that the global phase of a quantum state was physically irrelevant. The standard argument goes that all measurements in the quantum mechanical realm can be modelled as Hermitian operators ($\hat{O}$), which in turn can be represented in their spectral decomposition ($\hat{O} = \sum_{i} \omega_{i}\ket{\Psi_i}\bra{\Psi_i}$) which are themselves gauge-invariant with respect to gauge transformations of the type:

\begin{equation*}
  \mathcal{G}: \mathcal{H} \rightarrow \mathcal{H}^{*} \quad  \ket{\Psi} \mapsto e^{i\alpha(\Psi)}\ket{\Psi}
\end{equation*}

It would be naive to assume that phase is wholly irrelevant to quantum systems. It has been clear since the inception of quantum mechanics that this is not the case, and that the relative phase between two states can plays a crucial role in a system's behaviour. However, perhaps slightly more surprising is that the Hilbert space of a system, itself, shields certain internal phase relations within it from changing under arbitrary gauge transformations. It was Michael Berry in 1983 that discovered that quantum systems transported slowly around a circuit, by varying some control parameters $\bm{R}$, exhibit a phase factor which he dubbed the \textit{geometric phase} \cite{Berry1984}. In the same year Barry Simon made the connection between Berry's geometric phase and topology showing that Berry's phase is a result of the geometry of the parameter space of the control parameters ($\bm{R}$) over which the Hilbert Space is defined \cite{Simon1983}. With this discovery, the quantum world was intimately connected with the worlds of geometry and topology, and thus the gates to the study of quantum systems through this lens was opened.\\

This report will focus on only the beginning of this story; that being an introduction and exploration of the geometric phase, now more often called the Berry Phase\footnote{Following Berry's discovery of his self-dubbed geometric phase in 1983 and Simon's consequent generalization, further generalizations were made such as the introduction of The Aharonov-Anandan Phase in 1986 \cite{Aharonov1987}. As time progressed further connections were made to a variety of phase factors found in other fields of physics, all of which could be reasonably called a geometric phase. Using the term "geometric phase" for all such factors, while in of itself correct as they are all inherently connected by the same mathematical concepts, is rather unhelpful in helping an interested reader navigate the existing literature. As such, to avoid confusion, I will be using the term the Berry Phase, meaning the geometric phase factor discovered by Michael Berry in the context of the quantum physics.}. In this report I will begin by deriving the Berry Phase in a manner inherently intertwined with Simon's insights (Sec: \ref{ssec:discrete_case}, \ref{ssec:continuous_case}). This theoretical discussion will be taken one step further by examining the phase derived and introducing a few more closely related quantities (Sec: \ref{ssec:BP_Properties}). Following these purely mathematical discussions of the Berry Phase we will take a step over to physical theory and present a proof of the Berry Phase in the context of the Adiabatic Theorem, where Michael Berry himself discovered the phase (Sec: \ref{ssec:adiabatic_theorem}) then going on to detail the theoretical arguments through which Berry unveiled the Berry Phase. In the final section of the body of this report two physical systems will be explored in which the Berry Phase arises naturally; that of a Magnetic 1/2-spin-moment in a precessing magnetic field (Sec: \ref{ssec:magnetic_moment}) in which both an analytical as well as a numerical solution will be presented, as well as that of the world famous Aharonov-Bohm effect (Sec: \ref{ssec:aharonov_bohm_effect}).\\

I urge the reader to keep in mind that the relevance of this report derives itself from the importance of the much larger field of topological physics and the role the Berry Phase plays in connecting the quantum realm to the realm of topology. Understanding the Berry Phase in this context provides a natural jumping off point into a much wider world, with a whole array of interesting phenomena and insights.

\section{Theory\label{sec:the_berry_phase}}

\subsection{The Berry Phase}
\subsubsection{The Discrete Case \label{ssec:discrete_case}}

  \begin{wrapfigure}{l}{0.3\textwidth}
    \includegraphics[width=0.3\textwidth]{berry_loop}
    \caption{Diagram illustrating the Berry Phase sum in the discrete case.}
    \label{fig:berry_loop}
  \end{wrapfigure}

In this section the Berry Phase will be introduced in the discrete case; that is, it will be defined for a Hilbert space ($\mathcal{H}$) with a finite number of states ($N$). \\ 

Before introducing the Berry Phase we first have to define the relative phase between two states ($\ket{\Psi_1}$, $\ket{\Psi_2}$):

  \begin{equation*} \label{eq:relative_phase}
    \gamma_{1,2} = -arg \braket{\Psi_{1}|\Psi_{2}} \qquad \gamma_{1, 2} \in (-\pi, \pi]
  \end{equation*}

That being the phase factor of the complex number describing the inner product of the two states with one another. Here the total range of values allowed for the phase is restricted, respecting the equivalence of values outside this range under the relation $\gamma_{0} \equiv \gamma_{1} \Leftrightarrow e^{i\gamma_{0}} = e^{i\gamma_{1}}$. One can quickly show that this phase factor itself is not of any inherent physical relevance as it is expressly dependent on our initial choice of a gauge:
  \begin{equation*}
    \ket{\Psi_{j}} \rightarrow \exp(i \alpha_{j}) \ket{\Psi_{j}} \quad\quad
    \exp(-i\gamma_{1,2}) \rightarrow \exp(-i\gamma_{1,2} + i(\alpha_{2} - \alpha_{1}))\\
  \end{equation*}
Restricting ourselves now, for the sake of simplicity, to a finite Hilbert space (an $N$-state system) one can define the Berry Phase along any given \textit{loop} $c_m=[n_{0}, n_{1}, \ldots, n_{m}, n_{m+1}], \quad n_{0} = n_{m+1}$ (Fig: \ref{fig:berry_loop}):
  \begin{equation}
    \gamma_{L}(c_m) = \sum_{i \leq m}\gamma_{n_{i}, n_{i+1}}
  \end{equation}

Applying a general gauge transformation ($\mathcal{G}$) to the Hilbert space, in which our states live, we can utilize the transformation properties of the relative phase (Eq: \ref{eq:relative_phase}) to show that the Berry Phase over an arbitrary sequence defining the \textit{loop} (i.e. $c_m$) is gauge-invariant.
\begin{proof}
From the difference in the relative phase from its original value under an arbitrary gauge transformation
\begin{align*}
\gamma_{i, j} \rightarrow \gamma_{i, j} + \delta_{i, j} && \delta_{i, j} := (\alpha_i - \alpha_{j}) \; \mathrm{mod} \; 2\pi \\
\end{align*}
We can define the difference of the Berry Phase under the same transformation as the sum of these quantities over the loop
\begin{align*}
\gamma_{L}(c_m) \rightarrow \gamma_{L}(c_m) + \delta\gamma(c_m) && \delta\gamma(c_m) := \sum_{i=0}^{m}\delta_{i,i+1} 
\end{align*}
Furthermore, it holds trivially that $\delta_{i, j} + \delta_{j, k} = \delta_{i, k}$ and therefore the difference in the Berry Phase from its original value following the gauge transformation evaluates to zero.
\begin{equation*}
  \delta\gamma(c_m) = \sum_{i=0}^{m}\delta_{i,i+1} = \delta_{0,m + 1} = \delta_{0, 0} = 0
\end{equation*}
\end{proof}

\vspace{2ex}
This result is not altogether surprising (the proof itself is fairly straightforward), it is however an interesting result that a gauge-invariant quantity can be constructed from the sum of many non-gauge invariant quantities; that is that a number of quantities that by themselves have no physical meaning can combine to create a quantity which does. This de-facto \textit{1=0} result, begs the question if there is not another gauge-invariant formulation of the Berry Phase which does in fact consist of only non-gauge invariant components. This is in fact possible returning to the original representation for the relative phase between two states:

  \begin{align*}
    e^{-i\gamma_{1,2}} &\propto \braket{\Psi_1|\Psi_2}\\
    \Rightarrow e^{-i \gamma_L(c_m)} &\propto  
                         \prod_{i = 0}^{m}  \braket{\Psi_{i}|\Psi_{i+1}} = \Tr \left( \prod_i^m \ket{\Psi_{i}}\bra{\Psi_{i}}\right)
  \end{align*}
  Here we see that if we don't extract the relative phase information of each step in the sequence, then we can express the Berry Phase as the phase component of the trace of the product of a number of projection operators, which themselves are gauge-invariant \cite{Asboth2016}.


\subsubsection{The Continuous Case \label{ssec:continuous_case}}

The most intuitive way to extend the Berry Phase to the continuous regime is by analogy to the discrete case. This will not be conducted in the manner of a formal mathematical proof showing that the same concepts introduced in the discrete case (\ref{ssec:discrete_case}) can be extended to the continuum, for such a proof please refer to \cite{Asboth2016}.\\

In the discrete case there were three fundamental components that were not explicitly identified, though were the key ingredients in allowing us to define the Berry Phase. These were as follows:

\begin{itemize}
  \item The unique identifiers that we used to distinguish the various states within the Hilbert Space from one another. We will call the set of all identifiers the parameter space and denote this with $\mathbb{P}$. In the discrete case this set was given by the natural numbers $\mathbb{N}$.
  \item A unique mapping from the parameter space ($\mathbb{P}$) to the Hilbert space of our quantum system ($\mathcal{H}$). In the discrete case the wave function was given by $\Psi: \mathbb{N} \rightarrow \mathcal{H}, \quad n \mapsto \ket{\Psi_{n}}$. 
  \item A path. The path will be denoted with the variable $\mathcal{C}$. In the discrete case this was simply a sequence $[n_0, n_1, \ldots, n_m]$. 
\end{itemize}

By identifying these components we have broken the search for the Berry Phase in the continuous regime down into the identification of the appropriate analogues for each of components identified above:

\begin{itemize}

  \item The first of which, the Parameter Space, is implicit in the nature of our question "How does one extend the Berry Phase to continuous Hilbert Spaces?": an n-dimensional continuous parameter space ($\mathcal{M}$).
  \item The wave function follows, being simply a mapping from the elements of the parameter space to the Hilbert Space: $\Psi: \mathcal{M} \rightarrow \mathcal{H}, \quad \bm{R} \mapsto \ket{\Psi(\bm{R})}$

  \item A path in a continuous space can no longer be described by a finite sequence and this concept must be expanded to the continuous realm:
  $\mathcal{C}: [0, 1] \rightarrow \mathbb{P}, \quad t \mapsto \bm{R}(t)$
\end{itemize}

Having translated the components required to define the Berry Phase in the discrete case to the continuous realm, it is now possible to postulate an expression for this phase when working with a continuous Hilbert Space. In the discrete case the Berry Phase is given as the sum of the relative phases of each wave function connected by the path $\mathcal{C}$:

\begin{equation*}
\gamma(\mathcal{C})) = \sum_{i = 0}^{N} \gamma_{n_{i \; \mathrm{mod} \; N}, n_{i+1 \; \mathrm{mod} \; N}} 
\end{equation*}

In the continuum, since states live infinitely close together we have to consider all terms of the form $\lim_{dt \to 0}\gamma_{\mathcal{C}(t), \mathcal{C}(t + dt)}$ in the \textit{summation}. The continuous \textit{summation} of a given term is realized by an integral of the term divided by the measure of the space over which it is integrated.

\begin{equation}\label{eq:berry_phase_analog}
  \gamma(\mathcal{C}) = \lim_{dt \to 0} \int_0^1 \gamma_{\mathcal{C}(t), \mathcal{C}(t + dt)} dt
\end{equation}

The main advantage in this analogy is that we can see that the Berry Phase in a continuous Hilbert Space can be understood as the summation of the relative phases of infinitesimally close together states along a closed path, remaining fundamentally as it was in the discrete case.

\begin{table}
\centering
\begin{tabular}{||l|l|l||}
\hline
&\textbf{Discrete} & \textbf{Continuous}\\
\hline
\textbf{Parameter Space ($\mathbb{P}$)} & $\mathbb{N}_{0}$ &  $\mathcal{M}$\\
\hline
\textbf{Wave Function ($\Psi$)} & $\Psi: \mathbb{P} \rightarrow \mathcal{H}, \quad n \mapsto \ket{\Psi_{n}}$ & $\Psi: \mathbb{P} \rightarrow \mathcal{H}, \quad \bm{R} \mapsto \ket{\Psi(\bm{R})}$ \\
\hline
\textbf{Path} ($\mathcal{C}$) &  $[n_{0}, n_{1}, \ldots, n_{N-1}]$ & $\mathcal{C}: [0, 1) \rightarrow \mathcal{M}, \quad t \mapsto \bm{R}(t)$\\
\hline
\textbf{Berry Phase} ($\gamma(\mathcal{C}$)) & $\sum_{i = 0}^{N} \gamma_{n_{i \; \mathrm{mod} \; N}, n_{i+1 \; \mathrm{mod} \; N}}$ & $\lim_{dt \to 0} \int_0^1 \gamma_{\mathcal{C}(t), \mathcal{C}(t + dt)} dt$\\
\hline
\end{tabular}
\captionof{table}{Table summarizing the parallels between the Berry Phase and related components in the discrete and continuous regimes.}
\end{table}

While the form of the Berry Phase derived above (Eq: \ref{eq:berry_phase_analog}) is useful for relating it to the compact case it is not yet in its most compact form. Utilizing integration by substitution, the continuity of the parametrization of the path $\mathcal{C}$ and assuming that this path is parametrized with respect to the arc length, we can transform the integral to a closed path integral in the parameter space ($\mathcal{M}$).
\begin{equation*}
  \gamma(\mathcal{C}) = \lim_{d\bm{r} \to 0} \oint_{\mathcal{C}} \gamma_{\bm{R}, \bm{R}+d\bm{r}} \cdot d\bm{R}
\end{equation*}
For the limit to be appropriately defined in the parameter space we have to assure that we continue to take the relative phase of state pairs lying on the parametrized path:
\begin{align*}
d\bm{r} &\in M_{R} && M_{R} := \{\bm{r} \in \mathbb{M}|\; \bm{R} + \bm{r} \in \mathcal{C}([0, 1])\}\\
\end{align*}
The limit can be taken by utilizing how the relative phase information is encoded into the Hilbert Space and taking the Taylor Expansion:
  \begin{equation*}
    \begin{aligned}
      &\lim_{d\bm{r} \to 0} \exp(-i \gamma_{\bm{R}, \bm{R + d\bm{r}}}) = 
       \lim_{d\bm{r} \to 0}\frac{\braket{\Psi(\bm{R})|\Psi(\bm{R} + d\bm{r})}}
       {|\braket{\Psi(\bm{R})|\Psi(\bm{R} +d\bm{r})}|}&\\
      &1 - i \gamma_{\bm{R}, \bm{R} + d\bm{r}} = \bra{\Psi(\bm{R})}(\ket{\Psi(\bm{R})} + \nabla_{\bm{R}} \ket{\Psi(\bm{R})})&\\
      &\Rightarrow \gamma_{\bm{R}, \bm{R + d\bm{r}}} =
       i\braket{\Psi(\bm{R})|\nabla_{\bm{R}}|\Psi(\bm{R})}
    \end{aligned}
  \end{equation*}
  with this we arrive at the most compact form of the Berry Phase in a continuous Hilbert Space:
  \begin{equation}
    \gamma(\mathcal{C}) = \oint i \braket{\Psi(\bm{R})|\nabla_{\bm{R}}|\Psi(\bm{R})} d\bm{R}
  \end{equation}
  The integrand of the Berry Phase in this form is known as the Berry Connection \cite{Asboth2016}. 
  \begin{equation*}
    \mathcal{A} := i \braket{\Psi(\bm{R})|\nabla|\Psi(\bm{R})}
  \end{equation*}

\subsection{Properties of the Berry Phase\label{ssec:BP_Properties}}

Having derived the Berry Phase for the continuous case we can begin analysing some of its various properties. Recalling that the Berry Connection takes on the role of measuring the relative phase difference of infinitely close together quantum states, a natural first step would be to examine how this quantity behaves under gauge transformations.

  \begin{align}\label{eq:berry_connection_invariance}
    \nonumber \ket{\Psi(\bm{R})} &\rightarrow \exp(i \alpha(\bm{R})) \ket{\Psi(\bm{R})}\\
    \mathcal{A} &\rightarrow \mathcal{A} - \nabla \alpha(\bm{R})
  \end{align}

  As we would expect this is a non-gauge invariant property, the same result retrieved for the discrete analogue, the relative phase. However, following the same argumentation laid out in the discrete case, the next natural step is to see if we can find a formulation of the Berry Phase whose integrand is gauge-invariant. This is achieved by utilizing Stokes Theorem on our expression for the Berry Phase.

  \begin{align*}
    \gamma(\mathcal{C}) &= \oint \mathcal{A} \cdot d\bm{R}\\
    & = \sum_{\mu, \nu} \int_{\mathcal{S}(\mathcal{C})} \frac{1}{2} \partial_{\mu} \mathcal{A}_{\nu} - \partial_{\nu} \mathcal{A}_{\mu} dR^{\mu} \wedge dR^{\nu}&
  \end{align*}


The integrand arising in this expression defines a second order tensor known as the Berry Curvature.

  \begin{equation*}
    \Omega_{\mu, \nu}(\bm{R}) := \partial_{\mu} \mathcal{A}_{\nu} - \partial_{\nu} \mathcal{A}_{\mu}  
  \end{equation*}

Examining this quantity for gauge-invariance utilizing the transformation rules for the Berry Connection derived above (Eq: \ref{eq:berry_connection_invariance}) we retrieve:

       \begin{align*}
         &\ket{\Psi(\bm{R})} \rightarrow \exp(i\alpha(\bm{R}))\ket{\Psi(\bm{R})}\\
         &\Omega_{\mu, \nu}(\bm{R}) \rightarrow \partial_{\mu} (\mathcal{A}_{\nu}
         - \partial_{\nu} \alpha(\bm{R})) - \partial_{\nu}(\mathcal{A}_{\mu} -
         \partial_{\mu} \alpha(\bm{R})) = \Omega_{\mu, \nu}(\bm{R})
      \end{align*}
confirming the gauge-invariance of this property.


\subsection{The Adiabatic Theorem}\label{ssec:adiabatic_theorem}

Up until this point the Berry Phase has been introduced as a purely mathematical construct arising as a result of the geometry of continuous Hilbert Spaces (Sec: \ref{ssec:discrete_case}, \ref{ssec:continuous_case}). There is of course no fundamental guarantee that there is any physical system that simulates the transformations on a given state, which are necessary in order to translate this mathematical construct into a physically observable quantity. In this section we will discuss a fundamental example where this occurs known as The Adiabatic Theorem.\\

The Adiabatic Theorem is a statement pertaining to the \textit{slow} evolution of quantum systems, though adiabatic behaviour is a concept that extends well beyond the quantum realm. Adiabatic behaviour, in general, is the idea that so long as the environment of any given system changes \textit{slowly} enough (with respect to the systems internal dynamics) then the system will continue to exhibit predictable behaviour, determined solely by the state of the internal system and the instantaneous\footnote{The external \textit{instantaneous} parameters will of course always influence the internal system, the postulate of adiabatic behaviour is that one does not have to keep track of its' \textit{dynamics} in the appropriate limit.} parameter values of the environment \cite{Griffiths2017}. The Adiabatic Theorem tailors this postulate to the quantum realm; defining what exactly it means for the parameters defining a quantum system to evolve sufficiently slowly, and describing how exactly the quantum time evolution is effected by this slow change of parameters in the adiabatic limit. One formulation of the Adiabatic Theorem is as follows:\\

\begin{thm}
If $H(t)$, $0 \leq t \leq 1$ is  family of Hermitian Hamiltonians and if $E_n(t)$, $E_m(t)$ designate smooth functions of $t$ which are simple eigenvalues of $H(t)$ isolated from the rest of the spectrum, the state of a quantum system \[\Psi(t) = \sum_{n} c_{n}(t)\ket{n(t)}\] given by the time-dependent Schrödinger equation
  \[i\frac{d}{dt}\Psi_{T} = H(\frac{t}{T})\Psi_{T}(t)\]
evolves such that the amplitudes of the instantaneous eigenstates composing the wave function are given by:
    \begin{align*} 
        c_{n}(t) = c_{n}(0)e^{i\theta_{n}(t)}e^{i\gamma_{n}(t)} &&
        \theta_{n}(t)  &= -\hbar^{-1} \int_{0}^{t} E_{n}(t^{\prime}) dt^{\prime}\\  
        && \gamma_{n}(t) &= i\int_{0}^{t} \braket{n(t^{\prime})|\dot{n}(t^{\prime})}
        dt^{\prime}
    \end{align*}
as $T \to \infty$ \cite{Born1928, Berry1984, Simon1983, Sakurai1994}.
\end{thm}

\begin{proof}
First we can derive a differential equation for the amplitudes of the individual eigenstates utilizing the time-dependent Schrödinger equation
      \begin{align*}
        &i\hbar\ket{\dot{\Psi}(t)} = \bm{H}(\frac{t}{T})\ket{\Psi(t)} && \text{Schrödinger equation}&\\
        &i\hbar \sum_{n} \left[ \dot{c}_{n}(t)\ket{n(t)} + c_{n}(t)\ket{\dot{n}(t)} \right] = \sum_{n} c_{n}(t) E_{n}(t) \ket{n(t)}&& \text{Linear expansion}&\\
        &i\hbar \dot{c}_{m}(t) + i \hbar \sum_{n} \braket{m(t)|\dot{n}(t)}
          = c_{m}(t) E_{m}(t)&& \text{Inner product with $\bra{m(t)}$ (*)}
      \end{align*}
Then we can utilize the time-independent Schrödinger equation to retrieve an expression that helps to simplify the differential equation above.
      \begin{align*}
        &\bm{H}(\frac{t}{T})\Psi(t) = \sum_{n}c_{n}(t)E_{n}(t)\ket{\Psi_{n}(t)} && \text{time-independent S.E.} &\\
        &\frac{1}{T}\dot{\bm{H}}(t) \ket{n(t)} + \bm{H}(\frac{t}{T}) \ket{\dot{n}(t)} = \dot{E}_{n}(t) \ket{n(t)} + E_{n}(t) \ket{\dot{n}(t)}&& \text{Apply $\partial_{t}$} &\\
         &\braket{m(t)|\frac{1}{T}\dot{\bm{H}}(t)|n(t)} + E_{m} \braket{m(t)|\dot{n}(t)} 
         =  E_{n}(t) \braket{m(t)|\dot{n}(t)}&& \text{Inner product with $\bra{m(t)}$}&\\
         &\braket{m(t)|\dot{n}(t)} 
         = \frac{\braket{m(t)|\frac{1}{T}\dot{\bm{H}}(t)|n(t)}}{E_{n}(t) 
         - E_{m}(t)} \quad (m \neq n)&& \text{Rearrange (**)}&\\
       \end{align*}
Finally we insert the result from the time-independent equation into those of the time-dependent equation and utilize the assumptions of the theorem to solve the resulting differential equation.
      \begin{flalign*}
        &\dot{c}_{m}(t) + \left( \frac{i}{\hbar} E_{m}(t) + \braket{m(t)|\dot{m}(t)} \right) c_{m}(t) =  \sum_{\substack{n\\ n \neq m}} \frac{\braket{m(t)|\frac{1}{T}\dot{\bm{H}}(t)|n(t)}}{E_{m}(t) - E_{n}(t)}&& \text{(**) into (*)}&\\
        &\dot{c}_{m}(t)  
        = i \left( - \frac{1}{\hbar}E_{m}(t) + i\braket{m(t)|\dot{m}(t)} \right)
        c_{m}(t)&& \text{limit as $T \to \infty$} &\\
        &c_{m}(t) 
        = c_{m}(0) e^{-i \frac{1}{\hbar} \int_{0}^{t} E_{m}(t^{\prime}) dt^{\prime}}
        e^{i\int_{0}^{t} i\braket{m(t^{\prime})|\dot{m}(t^{\prime}}dt^{\prime}}&& \text{Solve} &\\ 
      \end{flalign*}\cite{Sakurai1994, Berry1984}
\end{proof}

In the above proof we see that the role of sufficient slowness and the concept of the adiabatic limit in the quantum realm is given by:
$\sum_{\substack{n\\ n \neq m}} \frac{\braket{m(t)|\frac{1}{T}\dot{\bm{H}}(t)|n(t)}}{E_{m}(t) - E_{n}(t)}$
In other words the timescale over which one must traverse a given family of Hamiltonians for the adiabatic approximation to hold, must be so long that at all time points the factor given by the sum is negligibly small with respect to the timescale of the evolution. The question remains however where does the geometric phase factor arise?\\

There are two phase factors that arise in this theorem: 

\begin{itemize}

  \item The first phase factor $\theta_{m}(t)  = -\hbar^{-1} \int_{0}^{t} E_{m}(t^{\prime}) dt^{\prime}$ is simply a result of the evolution of the Eigenenergy of the m-th state; initialization of the state in an eigenstate and solving the time-dependent Schrödinger equation retrieves this factor.

  \item The second phase factor $\gamma_{m}(t) = i\int_{0}^{t} \braket{m(t^{\prime})|\dot{m}(t^{\prime})} dt^{\prime}$ is less trivial then the \textit{dynamic phase} above and can be shown to be the Berry Phase. This is done by assuming that the time evolution of the Hamiltonian is mediated by a number of control parameters represented by some n-dimensional vector $\bm{R}$ 

  \begin{equation*}
    \bm{H}(t) = \bm{H}(\bm{R}(t)) \quad \Psi(t) = \Psi(\bm{R}(t))
  \end{equation*}

  \begin{flalign*}
    \gamma &=  \int_{0}^{t} i \braket{m(\bm{R}(t))|\dot{m}(\bm{R}(t))} dt&& \text{Specify dependency on $\bm{R}$}&\\
           &=  \int_{0}^{t}
    i \braket{m(\bm{R}(t))|\nabla_{\bm{R}}|m(\bm{R}(t))} \dot{\bm{R}}(t) dt && \text{Chain rule}&\\
           &= \oint i \braket{m(\bm{R})|\nabla_{\bm{R}}|m(\bm{R})}d\bm{R} && \text{Substitution}&\\
  \end{flalign*}

  Whereby we then recover the same expression for the Berry Phase that we derived in Section \ref{ssec:continuous_case}.
\end{itemize}

\section{Physical Systems}\label{sec:physical_systems}
\subsection{Magnetic Moment in a Precessing Field}\label{ssec:magnetic_moment}

While the Adiabatic Theorem is clearly broadly applicable to a broad range of quantum physical phenomena, and thus we can expect to find the Berry Phase in a wide range of systems, it is of course prudent to prescribe the theory discussed in the last section (Sec: \ref{ssec:adiabatic_theorem}) to a concrete system, for which we can then calculate the corresponding Berry Phase factor. To this end we will examine a spin-1/2 system under the influence of a magnetic field, changing in direction though not in magnitude.
  
A full description of the system is achieved by parametrizing the motion of the magnetic field in spherical coordinates and inserting it into the Hamiltonian describing the system.
      \begin{align*}
        \bm{B}(t) = B_{0} \begin{pmatrix} 
                                          \sin\theta(t)\cos\phi(t)\\
                                          \sin\theta(t)\sin\phi(t)\\
                                          \cos\theta(t)
                           \end{pmatrix}
        && \bm{H}(t) = \frac{\hbar e}{2m}\bm{B} \cdot \bm{\sigma}
      \end{align*}
The eigenstates of this Hamiltonian consist of the well known \textit{spin-up} and \textit{spin-down} wave functions for a magnetic field directed along a arbitrary axis in space. 
      \begin{align*}
        \chi_{+}(t) &= \begin{pmatrix} 
                                      \cos\frac{\theta(t)}{2}\\
                                      e^{i\phi(t)} \sin \frac{\theta(t)}{2}
                       \end{pmatrix} 
                       && E_{+} = \frac{\hbar \omega_{1}}{2}\\
       \chi_{-}(t) &= \begin{pmatrix} 
                                     e^{-i\phi(t)}\sin\frac{\theta(t)}{2}\\
                                     -\cos\frac{\theta(t)}{2}
                       \end{pmatrix} 
                       && E_{-} = -\frac{\hbar \omega_{1}}{2}
      \end{align*}

\begin{wrapfigure}{r}{0.3\textwidth}
  \label{fig:spin_system}
  \includegraphics[width=0.3\textwidth]{b_spin_system_2}
  \caption{Motion of a 1/2-spin in an adiabatically changing magnetic field \cite{Griffiths2017}.}
\end{wrapfigure}

We thus see that the control parameters defining the evolution of our quantum system are given by the spherical coordinates $r, \theta, \phi$ \cite{Griffiths2017}.\\

According to the Adiabatic Theorem (See: \ref{ssec:adiabatic_theorem}) the calculation of the end state of the system is given by simple calculation of the dynamic ($\theta$) and geometric ($\gamma$) phase factors of each eigenstate comprising the state in which our quantum system was initialized. Furthermore, if we evolve the quantum system adiabatically along a closed path then the geometric phase factor will correspond to the gauge-invariant Berry Phase. For this system we can then calculate the Berry Phase for an adiabatically changing system by direct integration of the Berry Connection along the path chosen in the spherical coordinate space:
  \begin{align}
    &\gamma = \int_{\mathcal{C}} i \braket{\chi_{+}|\nabla_{r, \theta, \phi}|\chi_{+}} \cdot
    d\bm{B}& \label{eq:magnetic_berry_phase}\\
    &\braket{\chi_{+}|\nabla_{r, \theta, \phi}|\chi_{+}} = i \frac{\sin^{2}(\theta /
        2)}{r \sin(\theta)} \hat{\phi}
  \end{align}
Or by integration of the corresponding Berry Curvature over the entire surface enclosed by the path.
  \begin{align}
    &\nonumber \gamma_{+} = \int_{\mathcal{S}(\mathcal{C})} i \nabla_{r, \theta, \phi} \times
     \braket{\chi_{+}|\nabla_{r, \theta, \phi}|\chi_{+}} \cdot d\bm{s}&\\
    &\nonumber \nabla_{r, \theta, \phi} \times
      \braket{\chi_{+}|\nabla_{r, \theta, \phi}|\chi_{+}} = \frac{i}{2r^{2}}\hat{r}&\\
    &\gamma = -\frac{\Omega}{2}
   \end{align}
   Doing so gives us a wonderful geometric interpretation of the Berry Phase as simply half of the solid angle enclosed by the path of the magnetic field vector in its evolution. The Berry Phase as a function of the solid angle has a periodicity of $4\pi$ which must give us the solid angle of the entire space, i.e. the solid angle associated with our system remaining at a singular point in parameter space. This result is of course unsurprising, as the solid angle of a spherical parameter space is a well known constant, but it is nonetheless a good practice in showing how information of the parameter space is embedded within the Berry Phase \cite{Griffiths2017}.\\

We can now use the result to see how the geometry of the parameter space embeds itself in the adiabatic evolution of a general state. Calculating the berry connection for the \textit{spin-down} state we find:
\begin{equation*}
  \braket{\chi_{-}|\nabla_{r, \theta, \phi}|\chi_{-}} = -i \frac{\sin^{2}(\theta /
  2)}{r \sin(\theta)} \hat{\phi}
\end{equation*}
and thus $\gamma_{-} = -\gamma_{+}$. Taking now a general state on the Bloch-sphere and evolving the system along a closed path results in a geometric phase factor for each eigenstate in the superposition. Rearranging then yields the transformation rule for a general state undergoing a cyclic adiabatic transformation:

\begin{align*}
  \Psi_i &= \cos \frac{\theta}{2} \ket{+} + e^{-i\phi}\sin \frac{\theta}{2}\ket{-}\\
  \Psi_f &= e^{-i\gamma_{+}}\cos \frac{\theta}{2} \ket{+} + e^{-i(\phi + \gamma_{-})}\sin \frac{\theta}{2}\ket{-}\\
  \Psi_f &= \cos \frac{\theta}{2} \ket{+} + e^{-i(\phi + \Omega)}\sin \frac{\theta}{2}\ket{-}\\
  \phi &\rightarrow \phi + \Omega
\end{align*}
Here we can see, so long as the solid angle enclosed by the path of the magnetic field vector is less then $2\pi$, then the resulting state will physically distinguishable from the initial one following the transformation.\\

\subsubsection{Numerical Solution}

\begin{figure}
  \begin{minipage}{.48\textwidth}
    \centering
    \label{fig:berry_path}
    \includegraphics[width=\textwidth]{berry_phase_path_90_90_-90_256.png}
    \caption{Example closed path of a magnetic field vector along the unit sphere.}
  \end{minipage}%
  \hspace{2ex}
  \begin{minipage}{.40\textwidth}
    \centering
    \includegraphics[width=\textwidth]{berry_phase_random_path.png}
    \caption{More complicated bath with a Berry Phase of $\gamma = 0.3609$ according to the numerical method}
    \label{fig:berry_path_random}
  \end{minipage}
\end{figure}

While an analytical solution of the Berry Phase of this system is possible over the Berry Curvature, complex paths will of course make defining the region over which to integrate increasingly difficult. For complex paths then, a numerical solution is of increasing necessity. For numerical calculation, however, the formulation of the Berry Phase as a path integral (Eq: \ref{eq:magnetic_berry_phase}) is more fitting, as calculation of the solid angle of the path would require the definition and classification of both the interior and exterior points of any given path on the sphere before calculation could commence (utilizing, for example, a Monte-Carlo method).\\

In both cases regardless, before calculation can begin, a path needs to be generated. In the program described here a given path is generated by discretizing the theta ($\theta$) and ($\phi$) coordinates in the regions mapping to the unit sphere (following a coordinate transformation back to a Cartesian representation).
\begin{equation*}
  \theta \in [0, \pi]  \qquad  \phi \in [0, 2\pi)
\end{equation*}
A path is then drawn on this grid starting from the North Pole ($(0,0)$) by successive user input. For each \textit{segment} of a path the user is prompted to choose a direction ($\hat{e}_{\theta}$, $\hat{e}_{\phi}$) along which they would like to travel as well as an angular distance ($\theta$, $\phi$), indicating how many degrees the user wishes to travel in the chosen direction. For each segment chosen by the user the program checks whether any previous segment shares a point with the current one, indicating an   intersection. If an intersection is found the input is refused, and the user is reminded that the path must meet up with the initial point ($(0,0)$) to form a closed path.\\

Following path generation the numerical calculation of the Berry Phase is straightforward, as the path points can be considered to be the points of a curve parametrized with respect to the arc length.
  \begin{equation*}
    \mathcal{C}: [0,1] \rightarrow \mathbb{S}^{2}
  \end{equation*}
Therefore the magnitude of the vector connecting consecutive points can be considered a first-order approximation of the distance between them, which corresponds to a first order approximation of the width of the interval between them in the continuous parametrization.
\begin{equation*}
  \forall x, y \in \mathcal{C}([0, 1]), \; x = C(t_x), \; y = C(t_y): \; L(C([t_x, t_y])) = \int_{t_{x}}^{t_{y}}dt = t_{y} - t_{x} \approx ||y - x||
\end{equation*}
This means that we can use a simple Riemann-Sum to approximate the Berry Phase (whereby $P$ denotes the set of points belonging to the discrete path)
\begin{equation*}
  \gamma_{+} = \sum_{i < |P|} \frac{sin^{2}(\theta_{i}/2)}{r_{i}\sin(\theta)}\hat{e}_{\phi_{i}} \cdot (\bm{x}_{i + 1} - \bm{x}_{i})
\end{equation*}
and that its not necessary to multiply by the interval width in the parameter space since this information is embedded in the vector $\bm{x}_{i + 1} - \bm{x}_{i}$.\\

We can check the accuracy of the numerical result by taking a simple path such as traversing from the \textit{North Pole} to the \textit{Equator}, then over 90 degrees in the azimuthal direction and back up to the starting position. Here the analytical result is (ignoring the sign).
\begin{equation*}
  \gamma = \frac{\pi}{4} \approx 0.78540
\end{equation*}
while the numerical result gives
\begin{equation*}
  \gamma = 0.79331
\end{equation*}
with discretized sphere represented by 256 equally spaced points in the interval $[0, \pi]$ for both the azimuthal and polar directions. This of course can be improved by increasing the number of points used to represent the sphere and thus decreasing the distance of the points along any given path.

\begin{table}[h]
  \centering
  \begin{tabular}{||l|l|l||}
    \hline
    \textbf{256-Points} & \textbf{1024-Points} & \textbf{8192-Points}\\
    \hline
    0.79331 & 0.78737 & 0.78564\\
    \hline
  \end{tabular}
\end{table}

\subsection{The Aharonov-Bohm Effect}\label{ssec:aharonov_bohm_effect}

While one of the most prominent examples of the Berry Phase arises during the adiabatic evolution of quantum systems, it would be naive to consider the Berry Phase an adiabatic effect. Looking back at theoretical discussions which kicked of this report in Section \ref{sec:the_berry_phase} it was clear that the Berry Phase arose due to internal phase relations of quantum states along a given path across the parameter space; the rate at which a given system underwent change did not enter into the discussion. One prominent effect where the Berry Phase arises and is not the result of the adiabatic change of a number of control parameters is the famous Aharonov-Bohm effect.\\

The Aharonov-Bohm effect was introduced in a paper by Yakir Aharonov and David Bohm in 1959. The paper put forward two physical systems showing that the electromagnetic potential was capable of acting upon particles independent of whether or not this corresponded with an observable electric or magnetic field. The two systems put forward came to be known as:

\begin{itemize}
  \item The electric Aharonov-Bohm effect, for the system demonstrating the physical effects of the electric scalar potential.
  \item The magnetic Aharonov-Bohm effect, for the system demonstrating the physical effects of the magnetic vector potential.
\end{itemize}

This paper's significance lay in the fact that it called into question the presumed physical irrelevance of the electromagnetic potentials, which up until then had been considered solely useful mathematical constructs for representing the electric and magnetic fields. If it were to be maintained that the physically relevant quantities remained to be those simply of the E and B-fields then the concept of locality itself would have to be called into question; A field present in one region of space must effect the behaviour of a particle in another region from which it is seemingly expelled. On the other hand if it is maintained that the electromagnetic potential is indeed the fundamental quantity in a physical system then one has to grapple with the fact that a quantity which exhibits a gauge-degree of freedom is somehow physical. It turns out that both of these effects can be elegantly explained as an example of the Berry Phase which we will show for each of these effects \cite{Aharonov1959}.

\subsubsection{The Electric Effect}\label{sssec:electric_effect}

 \begin{figure}[h]
   \centering
   \includegraphics[width=0.6\textwidth]{electric_effect}
   \captionof{figure}{Electric effect schematic \cite{Aharonov1959}.}
   \label{fig:ABE}
 \end{figure}
 
The system in which the Electric Aharonov-Bohm effect arises is described by two sufficiently long parallel paths down which a particle beam of electrons can travel (which we will refer to as the left and right paths and denote $L$ and $R$). Along each of these paths part of the path is shielded by the effects of an electric field by an idealized \textit{Faraday Cage}. One can consider a beam of electrons at some initial point being split into two parts and travelling separately along the left and right paths. Once the electron beams have arrived at their respective Faraday Cages local time dependent electric fields are turned on. Then before each of the particle beams exits their respective Faraday Cages the local electric fields are turned off. Finally, the beams are recombined at a later point (See: \ref{fig:ABE}).\\

For a quantum mechanical description of this system we can write down the Hamiltonian
\begin{equation*}
  \bm{H} = \bm{H}_{0} + V(\bm{r},t)
\end{equation*}

Whereby $\bm{H}_{0}$ describes our system in the absence of any electric fields. We can specify the system further by restricting our space to the symmetrically split free state solutions travelling along the predefined paths through the Faraday Cages (i.e. the particle beams, $\hat{\bm{X}}$ represents the reflection operator over the x-axis).
\begin{align*}
  \Psi_{0}(\bm{r},t)=\Psi_{L0}(x,t) + \Psi_{R0}(x,t) && \hat{\bm{X}}\Psi_{L0/R0} = \Psi_{R0/L0}
\end{align*}
In this restricted solution space the effect of two localized electric fields along each of the particle beam paths and one global field is indistinguishable. Therefore, we can replace the global field operator with two localized operators acting solely along each of the paths.
\begin{align*}
  \bm{H} = \bm{H}_{0} + V_{L}(x,t) + V_{R}(x,t) && V|_{L/R} = V_{L/R}
\end{align*}
We further restrict the Hilbert Space of our unperturbed Hamiltonian ($\bm{H}_{0}$) to those states who are completely contained within the Faraday Cage during the entire duration that there exists an electric field. For such states we can disregard the spatial dependence of the potential since the potential within the region of the Faraday Cage is constant.
\begin{equation*}
  \bm{H} = \bm{H}_{0} + V_{L}(t) + V_{R}(t)
\end{equation*}
Under these assumptions the solution of the time-dependent Schrödinger equation can be achieved by solving the equation independently for each of the paths where the Hamiltonian becomes
\begin{align*}
  \bm{H}|_{L} &= \bm{H}_{0} + V_{L}(x,t)\\
  \bm{H}|_{R} &= \bm{H}_{0} + V_{R}(x,t)
\end{align*}
and thus the solution of the time-dependent Schrödinger equation along each of the paths is given by
\begin{align*}
  \Psi_{L/R} &= \Psi_{0} \exp(-i \mathcal{S}_{L/R})\\
  \mathcal{S}_{L/R} &= \int_{0}^{t} V_{L/R}(t^{\prime}) dt^{\prime} 
\end{align*}
that is, the solution in the absence of any electric field ($\Psi_{0}$) times a phase factor given by the integral of the time dependent electric potential within the cage. It follows then that the quantum state of the entire system described by the Hamiltonian
  \begin{equation*}
    \bm{H}(\bm{r}, t) = \bm{H}_{0} + V(\bm{r}, t)
  \end{equation*}
can be written as for symmetrically split states whose wave functions are zero everywhere except within the Faraday Cages during the Electric Field event.
\begin{equation*}
  \Psi(\bm{r})=\Psi_{L0}(x,t)\exp(-i \mathcal{S}_{L}) + \Psi_{R0}(x,t)\exp(-i \mathcal{S}_{R})
\end{equation*}
Clearly the phase terms appearing in the solution will result in noticeable interference upon their recombination. Interestingly enough however one can calculate the Berry Phase of this wave function by integrating along one path and back along the other
\begin{align*}
  \gamma &= \oint i\braket{\Psi|\nabla_{t,x}|\Psi} \cdot d\bm{r}\\
         &= \int_{L} i\braket{\Psi_{L}|\nabla_{\bm{r}}|\Psi_{L}} + \int_{R} i\braket{\Psi_{R}|\nabla_{\bm{r}}|\Psi_{R}} \cdot d\bm{r}\\
         &= S_{L} - S_{R}
\end{align*}
Thus the electric effect and the resulting interference of the two wave functions can be attributed to the geometry of the resulting Hilbert Space being enforced upon the wave functions travelling along their respective paths \cite{Aharonov1959}.

\subsubsection{The Magnetic Effect}\label{sssec:magnetic_effect}

\begin{figure}[h]
  \centering
  \includegraphics[width=0.6\textwidth]{magnetic_effect}
  \captionof{figure}{Magnetic effect schematic \cite{Aharonov1959}.}
  \label{fig:AME}
\end{figure}

The system in which the magnetic effect arises is very much similar to that of the electric effect and consists of a split electron beam forced along paths wrapping around a solenoid containing a finite flux, and then recombined at some further point.\\

In general the Hamiltonian of such a system can be written as:
\begin{equation*}
  \bm{H} = \frac{\left[ \bm{P} - \frac{e}{c}\bm{A} \right]^{2}}{2m} \\
\end{equation*}
After which one can follow the same argumentation followed in the last section (Sec: \ref{sssec:electric_effect}); restricting our Hilbert space to those regions describing symmetric beams, and replacing the global vector potential operator $\bm{A}$ with two local operators acting solely on the two paths $\bm(A)_{L/R}$ in order come up with a solution of the Hamiltonian
\begin{align*}
  \Psi(\bm{r})&=\Psi_{L0}(x,t)\exp(-i \mathcal{S}_{L}) + \Psi_{R0}(x,t)\exp(-i \mathcal{S}_{R})\\
  \mathcal{S}_{L/R} &= \int \bm{A} \cdot d\bm{x} 
\end{align*}
Since the form of this solution is fundamentally the same as the form of the solution found in the electric effect, we see that here as well that the phase factor arising in this system is an example of the Berry Phase \cite{Aharonov1959}.
  \begin{equation*}
    \gamma = \oint i\braket{\Psi|\nabla_{\bm{r}}|\Psi} \cdot d\bm{r} = S_{L} - S_{R}
  \end{equation*}

While it is perhaps initially surprising that the Aharonov-Bohm effect is ultimately described by the Berry Phase, this correspondence is simply a result of a nearly one to one correspondence between the relationships tying together the Berry Connection, Berry Curvature and the Berry Phase and those of the magnetic vector potential, magnetic flux density, and the magnetic flux.

    \begin{minipage}{0.4\textwidth}
       \begin{align*}
         \textbf{Phase:}&\\
         \gamma(C) &= \oint \bm{\mathcal{A}} \cdot d\bm{R} \\
                   &= \int_{\mathcal{S}(\mathcal{C})} 
                      \nabla \times \bm{\mathcal{A}} \cdot d\bm{s}\\
                   &= \int_{\mathcal{S}(\mathcal{C})} \bm{\Omega} \cdot d\bm{s}
       \end{align*}
    \end{minipage}
    \hspace{0.05\textwidth}
    \begin{minipage}{0.4\textwidth}
      \begin{align*}
        \textbf{Flux:}&\\
        \Phi(C) &= \oint \bm{A} \cdot d\bm{R} \\
                &= \int_{\mathcal{S}(\mathcal{C})}
                   \nabla \times \bm{A} \cdot d\bm{s} \\
                &= \int_{\mathcal{S}(\mathcal{C})} \bm{B} \cdot d\bm{s}
      \end{align*}
    \end{minipage}\\

    What this means is that the effect of the magnetic vector potential and other associated quantities manifests itself in the geometry of the Hilbert Space describing the quantum system \cite{Cohen2019}.

\section{Conclusion}\label{sec:conclusion}

  \begin{wrapfigure}{r}{0.4\textwidth}
    \includegraphics[width=0.4\textwidth]{quantum_hall}
    \captionof{figure}{The quantum-hall resistance.}
  \end{wrapfigure}

We began this report with an intuitive derivation of the Berry Phase in the continuous realm, utilizing the discrete case as a jumping off point (Sec: \ref{ssec:discrete_case}, \ref{ssec:continuous_case}). Here we learned that the Berry Connection can be intuitively thought of as the relative phase of two quantum states infinitesimally close to one another in a continuous parameter space. Furthermore, we learned that, while initial formulations of The Berry Phase were in terms of a number of non-gauge invariant quantities, there also exists formulations for both the discrete and continuous cases which express the Berry Phase in terms of gauge-invariant quantities. Once this intuition was formed, we went on to prove the Adiabatic Theorem and thereby show that one of its prominent features was the emergence of the same phase (Sec: \ref{ssec:adiabatic_theorem}). This is of course, as we mentioned, exactly how Michael Berry uncovered this phase in 1983. Following these discussions we turned towards physical systems in order to strengthen our understanding as to how these phase factors can arise in real physical phenomena (Sec: \ref{ssec:magnetic_moment}, \ref{ssec:aharonov_bohm_effect}). To this end, we examined the adiabatic evolution of a spin-1/2 system in a precessing magnetic field (Sec: \ref{ssec:magnetic_moment}) both analytically and numerically, followed by the electric and magnetic Aharonov-Bohm effects (Sec: \ref{ssec:aharonov_bohm_effect}). These physical systems showed us one concrete application of the Adiabatic Theorem and one instance in which the Berry Phase arose although the Adiabatic Theorem did not necessarily hold true.\\


  While the scope of this report has indeed been broad there are a number of highly interesting and relevant topics that we did not touch on. The Berry Phase itself is of significant physical relevance for a large number of physical systems and applications. For example, the Berry Phase is fundamentally connected to explaining the transverse quantum hall conductance with the Berry Curvature arising in its expression  \cite{Cohen2019}.
  \begin{equation*}
    \sigma_{xy} = \frac{e^{2}}{\hbar} \sum_{n} \int \frac{dk}{2\pi} \Omega_{z}^{n}
  \end{equation*}
  On a more technical note, the Berry Phase is being considered in quantum computation as a candidate for the implementation of the CPHASE gate. An element of a one of the universal sets of quantum gates that can simulate all unitary transformations. The reason the Berry Phase stands to be such a promising candidate is that its phase is a property of the space itself and not of any one state with zero-measure. This in practice should make this implementation more robust to noise (Fig: \ref{fig:geo_qbit}) \cite{Thapa}.
    
    \begin{figure}
      \centering
      \includegraphics[width=0.3\textwidth]{geo_qbit}
      \captionof{figure}{Figure demonstrating the Berry Phase of 2-state system disturbed by noise\cite{Thapa}.}
      \label{fig:geo_qbit}
    \end{figure}

As mentioned at the beginning of this report, this topic derives its importance from it being one of the first connections of the quantum realm to topological physics. I would encourage any interested reader to read the rest of the papers produced in the course of the seminar, to which this report is attached, in order to get a fuller appreciation of the field to which The Berry Phase is intrinsically linked.

\newpage
\bibliography{Zotero}
\bibliographystyle{apalike}

\end{document}
